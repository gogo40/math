\documentclass[a4paper,10pt]{article}
\usepackage[utf8]{inputenc}
\usepackage[mathrsfs]{}

%opening
\title{Utilizando transformada Z para resolver um problema de recorrência}
\author{Péricles Lopes Machado}

\begin{document}

\maketitle

\begin{abstract}

Neste artigo, a transformada Z é utilizada na resolução do problema:

\begin{equation}
\begin{array}{lcl} 
F(0) & = & 0, \mbox { para } n \leq 0 \\
F(n) & = & \displaystyle \frac{1} {n^2} \displaystyle 
\sum_{0 \leq k < n} \left[ 
\displaystyle \sum_{0 \leq i < k} [1 + F(k)] +
1 + 
\displaystyle \sum_{k \leq i < n} [1 + F(n - 1 - k)]
\right], \mbox{ para } n > 0
\end{array}
\label{problema}
\end{equation}


\end{abstract}


\section{Resolução}

Simplificando-se a eq. (\ref{problema}):

\begin{equation}
\begin{array}{lcl} 
 n^2 F(n) & = & n + \sum_{0 \leq k < n} [ k + k F(k)]  + 
\sum_{0 \leq k < n} [ n - k + (n - k) F(n - 1 - k)] \\
n^2 F(n) & = & n + 
\sum_{0 \leq k < n} [ k] + \sum_{0 \leq k < n}[k F(k)]  + 
\sum_{0 \leq k < n} [ n - k] +\sum_{0 \leq k < n} [(n - k) F(n - 1 - k)] \\
n^2 F(n) & = &  n + \sum_{0 \leq k < n} [n] + 2 \sum_{0 \leq k < n}[k F(k)] \\
n^2 F(n) & = & (n^2 + n) + 2 \displaystyle \sum_{k \geq 0} k F(k) u(n - 1 - k)
\end{array}
\end{equation}


obtém-se:


\begin{equation}
n^2 F(n)  =  (n^2 + n)  + 2 \left[ nF(n) \right] * \left[ u(n - 1) \right],
\label{simplificado}
\end{equation}

onde $u(n) = 1, \mbox{ para } n \geq 0$ e $u(n) = 0, \mbox{ para } n < 0$.


Aplicando a transformada  $\mathcal{Z}$ em (\ref{simplificado}) e fazendo $ Y = \mathcal{Z} \{ F(n) \} $:

$$
 z^2 \displaystyle \frac{d^2}{dz^2}Y = \mathcal{Z} \left\{ n^2 + n  \right\} - 2 z  
 \displaystyle \frac{d}{dz}Y \mathcal{Z} \left\{u(n-1)\right\}
$$

$$
\begin{array}{lcl} 
 z^2 \displaystyle \frac{d^2}{dz^2}Y = \mathcal{Z} \left\{ n^2 + n \right\} - 2 z   
 \left( \displaystyle \frac{z^{-1}}{1 - z^{-1}} \right) \displaystyle \frac{d}{dz}Y \\
 z^2 \displaystyle \frac{d^2}{dz^2}Y = \mathcal{Z} \left\{ n^2  \right\} + 
 \mathcal{Z} \left\{ n \right\} -  \left( \displaystyle \frac{2z}{z - 1} \right) \displaystyle \frac{d}{dz}Y
\end{array}  
$$

$$
\begin{array}{lcl} 
z^2 \displaystyle \frac{d^2}{dz^2}Y = \frac{z^{-1} (1 + z^{-1})}{(1 - z^{-1})^3} + 
 \frac{z^{-1}}{(1 - z^{-1})^2} -  \left( \displaystyle \frac{2z}{z - 1} \right) \displaystyle \frac{d}{dz}Y \\
 z^2 \displaystyle \frac{d^2}{dz^2}Y = \frac{z^2 (1 + z^{-1})}{(z - 1)^3} + 
 \frac{z}{(z - 1)^2} -  \left( \displaystyle \frac{2z}{z - 1} \right) \displaystyle \frac{d}{dz}Y \\
 z^2 \displaystyle \frac{d^2}{dz^2}Y = \frac{z(z + 1)}{(z - 1)^3} + 
 \frac{z}{(z - 1)^2} -  \left( \displaystyle \frac{2z}{z - 1} \right) \displaystyle \frac{d}{dz}Y \\
 z^2 \displaystyle \frac{d^2}{dz^2}Y = \frac{z(z + 1)}{(z - 1)^3} + 
 \frac{z(z - 1)}{(z - 1)^3} -  \left( \displaystyle \frac{2z}{z - 1} \right) \displaystyle \frac{d}{dz}Y \\
 z^2 \displaystyle \frac{d^2}{dz^2}Y = \frac{2z^2}{(z - 1)^3} -  \left( \displaystyle \frac{2z}{z - 1} \right) \displaystyle \frac{d}{dz}Y \\
 \displaystyle \frac{d^2}{dz^2}Y = \frac{2}{(z - 1)^3} -  \left( \displaystyle \frac{2}{z(z - 1)} \right) \displaystyle \frac{d}{dz}Y \\
\end{array}  
$$

Por fim,

\begin{equation}
 \displaystyle \frac{d^2}{dz^2}Y = \frac{2}{(z - 1)^3} -  \left( \displaystyle \frac{2}{z(z - 1)} \right) \displaystyle \frac{d}{dz}Y.
 \label{equacao.diferencial}
\end{equation}

Resolvendo a equação diferencial (\ref{equacao.diferencial}), usando o $\it{Wolfram Alpha}$:

\begin{equation}
\begin{array}{lcl} 
Y(z) = (c_1 z^2+ \\
2 \log(1-z) ((c_1-2) z-c_1+z^2-2 (z-1) \log(z))-\\
c_1 z-c_1+2 (-2 (z-1) Li_2(z)+z^2 (-\log(z))+z+\\
(z-1) \log^2(1-z)+\\
2 z \log(z)-3))/(z-1)+\\
c_2
\end{array}  
\label{solucao.eq.diferencial}
\end{equation}

Expandindo a eq. (\ref{solucao.eq.diferencial}):

\begin{equation}
\begin{array}{lcl} 
 Y(z) = \\
 \displaystyle \frac{c_1 z^2}{z-1}-\displaystyle \frac{c_1 z}{z-1}- \displaystyle \frac{c_1}{z-1}+\\
 \displaystyle \frac{2 c_1 z \log(1-z)}{z-1}-\displaystyle \frac{2 c_1 \log(1-z)}{z-1}+\\
 c_2-\\
 \displaystyle \frac{4 z Li_2(z)}{z-1}+\displaystyle \frac{4 Li_2(z)}{z-1}+\\
 \displaystyle \frac{2 z^2 \log(1-z)}{z-1}-\displaystyle \frac{2 z^2 \log(z)}{z-1}+\\
 \displaystyle \frac{2 z}{z-1}-\\
 \displaystyle \frac{6}{z-1}+\\
 \displaystyle \frac{2 z \log^2(1-z)}{z-1}-\displaystyle \frac{2 \log^2(1-z)}{z-1}-\\
 \displaystyle \frac{4 z \log(1-z)}{z-1}-\\
 \displaystyle \frac{4 z \log(1-z) \log(z)}{z-1}+\\
 \displaystyle \frac{4 \log(1-z) \log(z)}{z-1} + \\
  \displaystyle \frac{4 z \log(z)}{z-1}
\end{array}
\label{solucao.eq.diferencial.expandida}
\end{equation}

Simplificando a eq. (\ref{solucao.eq.diferencial.expandida}):

$$
\begin{array}{lcl} 
 Y(z) = \\
 c_1z - \displaystyle \frac{c_1}{z-1}+ 2 c_1  \log(1-z)+ c_2-\\
 4 Li_2(z)\left[\displaystyle \frac{z + 1}{z-1}\right]+\\
 \displaystyle \frac{2 z^2 }{z-1}\displaystyle \left[\log(1-z) - \log(z)\right]+\\
 \displaystyle \frac{2 z}{z-1}-\\
 \displaystyle \frac{6}{z-1}+\\
 4 \log(1-z)\log(z) \left[\displaystyle \frac{ z  + 1}{z-1}\right]+\\
 2 \log^2(1-z) - \\
 4 \log(1-z) \left[\displaystyle \frac{ z }{z-1}\right]-\\
 4 \log(z)\left[\displaystyle \frac{ z }{z-1}\right]
\end{array}
$$

Reorganizando:

\begin{equation}
\begin{array}{lcl} 
 Y(z) = \\
 c_1z - \displaystyle \frac{c_1}{z-1}+ 2 c_1  \log(1-z)+ c_2-\\
 4 Li_2(z)\left[\displaystyle \frac{z + 1}{z-1}\right]+\\
 \displaystyle \frac{2 z^2 - 4z}{z-1}\displaystyle \left[\log(1-z) - \log(z)\right] -\\
 \displaystyle \frac{2 z - 6}{z-1}+\\
 4 \log(1-z)\log(z) \left[\displaystyle \frac{ z  + 1}{z-1}\right]+\\
 2 \log^2(1-z)
\end{array} 
\label{solucao.eq.diferencial.simplificado}
\end{equation}


Aplicando a transformada  inversa $\mathcal{Z}^{-1}$ na eq. (\ref{solucao.eq.diferencial.simplificado}):


$$
\begin{array}{lcl} 
 x[n] = \\
 \mathcal{Z}^{-1}\left\{c_1z\right\} - 
 \mathcal{Z}^{-1}\left\{\displaystyle \frac{c_1}{z-1}\right\} + 
 \mathcal{Z}^{-1}\left\{2 c_1  \log(1-z)\right\} + 
 \mathcal{Z}^{-1}\left\{c_2\right\}-\\
 \mathcal{Z}^{-1}\left\{4 Li_2(z)\left[\displaystyle \frac{z + 1}{z-1}\right]\right\}+\\
 \mathcal{Z}^{-1}\left\{\displaystyle \frac{2 z^2 - 4z}{z-1}\displaystyle \left[\log(1-z) - \log(z)\right]\right\} -\\
 \mathcal{Z}^{-1}\left\{\displaystyle \frac{2 z - 6}{z-1}\right\}+\\
 \mathcal{Z}^{-1}\left\{4 \log(1-z)\log(z) \left[\displaystyle \frac{ z  + 1}{z-1}\right]\right\}+\\
 \mathcal{Z}^{-1}\left\{2 \log^2(1-z)\right\}
\end{array} 
$$


\end{document}
