\documentclass[a4paper,10pt]{article}
\usepackage[utf8]{inputenc}
\usepackage[mathrsfs]{}

%opening
\title{Utilizando transformada Z para resolver um problema de recorrência}
\author{Péricles Lopes Machado}

\begin{document}

\maketitle

\begin{abstract}

Neste artigo, a transformada Z é utilizada na resolução do problema:

\begin{equation}
\begin{array}{lcl} 
F(0) & = & 0, \mbox { para } n \leq 0 \\
F(n) & = & \displaystyle \frac{1} {n^2} \displaystyle 
\sum_{0 \leq k < n} \left[ 
\displaystyle \sum_{0 \leq i < k} [1 + F(k)] +
1 + 
\displaystyle \sum_{k \leq i < n} [1 + F(n - 1 - k)]
\right], \mbox{ para } n > 0
\end{array}
\label{problema}
\end{equation}


\end{abstract}


\section{Resolução}

Simplificando-se a eq. (\ref{problema}):

\begin{equation}
\begin{array}{lcl} 
 n^2 F(n) & = & n + \sum_{0 \leq k < n} [ k + k F(k)]  + 
\sum_{0 \leq k < n} [ n - k + (n - k) F(n - 1 - k)] \\
n^2 F(n) & = & n + 
\sum_{0 \leq k < n} [ k] + \sum_{0 \leq k < n}[k F(k)]  + 
\sum_{0 \leq k < n} [ n - k] +\sum_{0 \leq k < n} [(n - k) F(n - 1 - k)] \\
n^2 F(n) & = &  n + \sum_{0 \leq k < n} [n] + 2 \sum_{0 \leq k < n}[k F(k)] \\
n^2 F(n) & = & (n^2 + n) u(n - 1) + 2 \displaystyle \sum_{k \geq 0} k F(k) u(n - 1 - k)
\end{array}
\end{equation}


obtém-se:


\begin{equation}
n^2 F(n)  =  (n^2 + n) u(n - 1) + 2 \left[ nF(n) \right] * \left[ u(n - 1) \right],
\label{simplificado}
\end{equation}

onde $u(n) = 1, \mbox{ para } n \geq 0$ e $u(n) = 0, \mbox{ para } n < 0$.


Aplicando a transformada  $\mathcal{Z}$ em (\ref{simplificado}) e fazendo $ Y = \mathcal{Z} \{ F(n) \} $:

$$
 z^2 \displaystyle \frac{d^2}{dz^2}Y = \mathcal{Z} \left\{(n^2 + n) u(n - 1) \right\} - 2 z  
 \displaystyle \frac{d}{dz}Y \mathcal{Z} \left\{u(n-1)\right\}
$$

$$
\begin{array}{lcl} 
 z^2 \displaystyle \frac{d^2}{dz^2}Y = \mathcal{Z} \left\{(n^2 + n) u(n - 1) \right\} - 2 z   
 \left( \displaystyle \frac{z^{-1}}{1 - z^{-1}} \right) \displaystyle \frac{d}{dz}Y \\
 z^2 \displaystyle \frac{d^2}{dz^2}Y = \mathcal{Z} \left\{ n^2 u(n-1) \right\} + 
 \mathcal{Z} \left\{ n u(n - 1) \right\} -  \left( \displaystyle \frac{2z}{z - 1} \right) \displaystyle \frac{d}{dz}Y
\end{array}  
$$

$$
\begin{array}{lcl} 
z^2 \displaystyle \frac{d^2}{dz^2}Y = \frac{z^{-2} (1 + z^{-1})}{(1 - z^{-1})^3} + 
 \frac{z^{-2}}{(1 - z^{-1})^2} -  \left( \displaystyle \frac{2z}{z - 1} \right) \displaystyle \frac{d}{dz}Y \\
 z^2 \displaystyle \frac{d^2}{dz^2}Y = \frac{z (1 + z^{-1})}{(z - 1)^3} + 
 \frac{1}{(z - 1)^2} -  \left( \displaystyle \frac{2z}{z - 1} \right) \displaystyle \frac{d}{dz}Y \\
 z^2 \displaystyle \frac{d^2}{dz^2}Y = \frac{(z + 1)}{(z - 1)^3} + 
 \frac{1}{(z - 1)^2} -  \left( \displaystyle \frac{2z}{z - 1} \right) \displaystyle \frac{d}{dz}Y \\
 z^2 \displaystyle \frac{d^2}{dz^2}Y = \frac{(z + 1)}{(z - 1)^3} + 
 \frac{z - 1}{(z - 1)^3} -  \left( \displaystyle \frac{2z}{z - 1} \right) \displaystyle \frac{d}{dz}Y \\
 z^2 \displaystyle \frac{d^2}{dz^2}Y = \frac{2z}{(z - 1)^3} -  \left( \displaystyle \frac{2z}{z - 1} \right) \displaystyle \frac{d}{dz}Y \\
 \displaystyle \frac{d^2}{dz^2}Y = \frac{2z}{z^2(z - 1)^3} -  \left( \displaystyle \frac{2z}{z^2(z - 1)} \right) \displaystyle \frac{d}{dz}Y \\
\end{array}  
$$

Por fim,

\begin{equation}
 \displaystyle \frac{d^2}{dz^2}Y = \frac{2}{z(z - 1)^3} -  \left( \displaystyle \frac{2}{z(z - 1)} \right) \displaystyle \frac{d}{dz}Y.
 \label{equacao.diferencial}
\end{equation}

Resolvendo a equação diferencial (\ref{equacao.diferencial}), usando o $\it{Wolfram Alpha}$:

\begin{equation}
\begin{array}{lcl} 
Y(z) = (c_1 z^2+2 log(1-z) ((c_1-2) z-c_1+z^2-2 (z-1) log(z))-c_1 z \\
 -c_1-4 (z-1) Li_2(z) \\
 -2 z^2 log(z)+2 z+2 (z-1) log^2(1-z)\\
 +4 z log(z)-7)/(z-1)+c_2
\end{array}  
\label{solucao.eq.diferencial}
\end{equation}

Expandindo a eq. (\ref{solucao.eq.diferencial}):

\begin{equation}
\begin{array}{lcl} 
 Y(z) = (c_1 z^2)/(z-1)-\\
 (c_1 z)/(z-1)-\\
  c_1/(z-1)+\\
 (2 c_1 z log(1-z))/(z-1)-\\
 (2 c_1 log(1-z))/(z-1)+\\
 c_2-\\
 (4 z Li_2(z))/(z-1)+\\
 (4 Li_2(z))/(z-1)+\\
 (2 z^2 log(1-z))/(z-1)-\\
 (2 z^2 log(z))/(z-1)+\\
 (2 z)/(z-1)-7/(z-1)+\\
 (2 z log^2(1-z))/(z-1)-\\
 (2 log^2(1-z))/(z-1)-\\
 (4 z log(1-z))/(z-1)-\\
 (4 z log(1-z) log(z))/(z-1)+\\
 (4 z log(z))/(z-1)+\\
 (4 log(1-z) log(z))/(z-1)
\end{array}
\end{equation}



\end{document}
